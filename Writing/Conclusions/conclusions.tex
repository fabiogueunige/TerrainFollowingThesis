%%%%%%%%%%%%%%%%%%%%%%%%%%%%%%%%%%%%%%%%%%%%%%%%%%%%%%%%%%%%%%%%%%%%%%%%%%%%%%%%
%2345678901234567890123456789012345678901234567890123456789012345678901234567890
%        1         2         3         4         5         6         7         8
% THESIS CONCLUSIONS
\def\baselinestretch{1}
\chapter{Conclusions}
\label{chap:conclusions}
\ifpdf
    \graphicspath{{Conclusions/Figures/PNG/}{Conclusions/Figures/PDF/}{Conclusions/Figures/}}
\else
    \graphicspath{{Conclusions/Figures/EPS/}{Conclusions/Figures/}}
\fi
\def\baselinestretch{1.66}

Write the conclusions here...
% Concentrarsi su:
% 1) Riepilogo dei risultati ottenuti
% 2) Implicazioni per la comunità scientifica e per l'industria
% 3) Trovare vantaggi e svantaggi del metodo per l'industria (es: Mappatura batimetrica di precisione, 
% Ispezione e manutenzione subacquea, Monitoraggio ambientale, Ricerca oceanografica e archeologia subacquea)
% 4) Limitazioni dello studio
% 5) Prospettive future e direzioni di ricerca

% Questo approccio non solo migliora le prestazioni operative di un AUV, ma apre la strada a missioni più lunghe, 
% sicure ed efficienti, con un impatto significativo in applicazioni scientifiche, industriali e ambientali.