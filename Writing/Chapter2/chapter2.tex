\chapter{AUV Dynamic}
\label{chap:second}
\ifpdf
    \graphicspath{{Chapter2/Figures/PNG/}{Chapter2/Figures/PDF/}{Chapter2/Figures/}{Chapter2/Figures/EPS/}}
\else
    \graphicspath{{Chapter2/Figures/EPS/}{Chapter2/Figures/}}
\fi
% Before a fast overview
This chapter introduces the equations that describe the dynamics of an underwater vehicle. As a first step, it is essential to present 
the conventions used to define reference frames in marine robotics. This makes it possible to describe the general equations of dynamics 
\cite{fossenHandbookMarineCraft2011} for any marine robot in 6-DOF, degrees of freedom, and subsequently conduct a more in-depth analysis 
for the BlueRov2 class and the parameters used in the simulation.

The study begins with the fundamental equation for underwater dynamics. The simplifications due to low-speed conditions, together with the most relevant 
parameters, will be illustrated in detail.
Finally, the parameters adopted in the BlueRov2 simulation are derived from an identification study conducted by the Dynamical Systems and Ocean Robotics group (DSOR) 
of Instituto de Sistemas e Robótica from Instituto Superior Técnico (ISR-IST).

% dire che nel capitolo vengono mostrate le convenzioni per la robotica marina e la dinamica degli AUV
% vengono descritte le equazioni della dinamica nei 6DOF e successivamente 
\section{Reference frames and naming conventions}
% segui tesi 
To tackle the study of AUV dynamics, it is essential to define the main reference frames in accordance with marine robotics conventions, namely 
the body-fixed frame, where the dynamic of the vehicle is described, and an earth-fixed frame, with respect to the position and the orientation 
of the vehicle are described.

The body-fixed frame $\{B\}$ is a right-handed coordinate system, it is rigidly attached to the vehicle, with its origin $O_B$ located at the 
center of gravity (CG). The axes $\{x_B , y_B , z_B\}$ are defined following the ``SNAME`` notation
 \cite{thesocietyofnavalarchitectsandmarineengineerssnameNomenclatureTreatingMotion1950}, so:
 \begin{itemize}
    \item the $x_B$ axis points towards the head of the vehicle (longitudinal axis);
    \item the $y_B$ axis points to the right side of the vehicle (transverse axis);
    \item the $z_B$ axis points downwards (normal axis).
 \end{itemize}

Having defined the robot's frame of reference, we now need to define a fixed observation reference to estimate the position and movement of the AUV. 
We can therefore assume the rotation and curvature of the Earth as zero, given the low-speed and small variations in latitude and longitude operating conditions. 
Based on these considerations, we define a earth-fixed-frame $\{I\}$ at a point on the sea surface within the vehicle's area of operation. Following convention, 
the frame will be a local NED (North-East-Down) frame, then a right-handed coordinate system, with the axes pointing respectively:
\begin{itemize}
    \item the $x_I$ axis pointing towards the North;
    \item the $y_I$ axis pointing towards the East;
    \item the $z_I$ axis pointing downwards.
\end{itemize}
Given the assumption stated earlier, $\{I\}$ can be considered an inertial frame, so Newton's laws of motion are valid.

Having defined the two reference frames, it is possible to introduce the SNAME notation for all parameters essential for determining the position, orientation 
and velocity of the AUV, considering external forces and moments applied to it. The position of the origin of the body-fixed frame with respect to the inertial 
frame is defined through the vector $\mathbf{\eta}_1 = [x, y, z]^T$, while the orientation of $\{B\}$ with respect to $\{I\}$ is given by the rotation matrix 
$^{I}_{B}R$ defined by the Euler angles contained in the vector $\mathbf{\eta}_2 = [\phi, \theta, \psi]^T$.
The velocity of $\{B\}$ with respect to $\{I\}$, on the other hand, is divided into linear $\mathbf{\nu}_1 = [u, v, w]^T$ and angular $\mathbf{\nu}_2 = [p, q, r]^T$ 
components, while the external forces and moments are expressed in $\{B\}$ and are described by the vector $\mathbf{\tau} = [\mathbf{\tau}_1, \mathbf{\tau}_2]^T$, 
where $\mathbf{\tau}_1 = [X, Y, Z]^T$ for forces and by the vector $\mathbf{\tau}_2 = [K, M, N]^T$ for moments.\\

This notation is more easily visible in the table \ref{tab:Sname_param}, where the parameters that will be widely used in this thesis are defined in the 6 degrees of freedom.

\begin{table}[H]
    \centering
    \resizebox{0.95\textwidth}{!}{
        \begin{tabular}{SSSSS} \toprule
            {DOF} & {Direction} & {Position and Euler angles} & {Velocity} & {Force and Moment}  \\ \midrule
            {1} & {along $x_B$} & $x$  & {surge speed } $u$ & $X$  \\
            {2} & {along $y_B$} & $y$  & {sway speed } $v$  & $Y$   \\
            {3} & {along $z_B$} & $z$  & {heave speed } $w$  & $Z$  \\
            {4} & {rotation about $x_B$} & {roll angle } $\phi$ & {roll rate } $p$  & $K$   \\
            {5} & {rotation about $y_B$} & {pitch angle } $\theta$  & {pitch rate } $q$   & $M$  \\
            {6} & {rotation about $z_B$} & {yaw angle }  $\psi$ & {yaw rate } $r$  & $N$  \\ \bottomrule
        \end{tabular}
    }
    \caption{SNAME nomenclature and symbols \cite{abreuSensorbasedFormationControl2014}.}
    \label{tab:Sname_param}
\end{table}

\section{Dynamics}
% descrivere le equazioni della dinamica nei 6DOF
The dynamic equations of motion describe how forces and torques affect the movement of the vehicle. These equations are commonly expressed in the body-fixed 
reference frame, as this formulation keeps the inertia tensor constant and allows external forces (weight, buoyancy, hydrodynamic effects) to be represented more conveniently.

The derivation of the rigid-body dynamics follows Newton-Euler laws for both translational and rotational motion \cite{fossenHandbookMarineCraft2011}:
\begin{equation}
    \begin{cases}
        \sum F_{RB} = m[\mathbf{\nu}_2 \times \mathbf{\nu}_1 + \dot{\mathbf{\nu}}_1] \\
        \sum N_{RB} = I_{RB}\dot{\mathbf{\nu}}_2 + \mathbf{\nu}_2 \times I_{RB}\mathbf{\nu}_2
    \end{cases}
    \label{eq:newton_laws}
\end{equation}
where $m$ is the mass of the vehicle, $I_{RB}$ is the inertia tensor, $\sum F_{RB}$ are the external forces and $\sum N_{RB}$ are the external moments acting on the vehicle.\\
The equations \ref{eq:newton_laws} can be rewritten in a more compact matrix form as:
\begin{equation}
    M_{RB}\dot{\mathbf{\nu}} + C_{RB}(\mathbf{\nu})\mathbf{\nu} = \mathbf{\tau}
    \label{eq:6dof_dyn_simple}
\end{equation}
$M_{RB}$ is the rigid body inertia matrix, while $C_{RB}$ contains the Coriolis and centrifugal terms.
These matrices satisfy some key properties:
\begin{itemize}
    \item $\dot M_{RB} = 0$: the inertia matrix is constant in the body-fixed frame;
    \item $\dot M_{RB}^T = \dot M_{RB}$: the inertia matrix is symmetric and positive-definite. Moreover, when the body-fixed frame is centered at the 
    center of gravity and aligned with the principal axes of inertia, $M_{RB}$ is diagonal;
    \item $C_{RB}(\mathbf{\nu}) = -C_{RB}(\mathbf{\nu})^T$: the Coriolis and centripetal matrix can be parameterized to be skew-symmetric.
\end{itemize}
In this case, with the body-fixed frame centered at the center of gravity, the inertia matrix and the Coriolis matrix correspond to:
\begin{equation}
    \scalebox{0.95}{
        $ M_{RB} = diag(m, m, m, I_x, I_y, I_z) \qquad
        C_{RB}(\mathbf{\nu}) = \begin{bmatrix}
            mS(\mathbf{\nu}_2) & 0_{3 \times 3} \\
            0_{3 \times 3} & -S(I_{RB}\mathbf{\nu}_2)    
        \end{bmatrix} $
    }
    \label{eq:RB_matrices}
\end{equation}

The subscript $RB$ highlights that the formulation includes only rigid-body dynamics, with all external forces and moments grouped in the generalized vector
$\mathbf{\tau}_{RB} = [X, Y, Z, K, M, N]^T$. In order to account for different contributions, this term can be decomposed as:
\begin{equation}
    \mathbf{\tau}_{RB} = \mathbf{\tau} + \mathbf{\tau}_{A} + \mathbf{\tau}_{D} + \mathbf{\tau}_{R} + \mathbf{\tau}_{dist}
    \label{eq:6dof_tau_simple}
\end{equation}
where:
\begin{itemize}
    \item $\mathbf{\tau}$ represents control inputs given by the thrusters;
    \item $\mathbf{\tau}_{A}$ accounts for added mass and added Coriolis effects, described by the matrices $M_A$ and $C_A(\mathbf{\nu})$ respectively. 
    \begin{equation}
        \scalebox{0.9}{
            $ M_A = -diag(X_{\dot{u}}, Y_{\dot{v}}, Z_{\dot{w}}, K_{\dot{p}}, M_{\dot{q}}, N_{\dot{r}}) \quad
            C_A(\mathbf{\nu}) = \begin{bmatrix}
            0_{3 \times 3} & -S(M_A\mathbf{\nu}_1) \\
            -S(M_A\mathbf{\nu}_1) & -S(M_A\mathbf{\nu}_2)
            \end{bmatrix} $
        }
        \label{eq:added_mass}
    \end{equation}
    It can be studied by computing the kinetic energy imparted by the vehicle to the surrounding displaced fluid (even for inviscid fluid);
    \item $\mathbf{\tau}_{D}$ models drag forces and moments, can be represented by inverting the sign with 
    $D(\mathbf{\nu}) = D_1 + D_2(\mathbf{\nu})$, where $D_1$ is a linear damping matrix, while $D_2(\mathbf{\nu})$ is a quadratic damping matrix. 
    These terms account for hydrodynamic resistance, including contributions from skin friction and pressure-induced drag;
    \item $\mathbf{\tau}_{R}$ represents restoring forces and moments arising from buoyancy and weight imbalance. Defining $\mathbf{r}_b = [x_b, y_b, z_b]^T$ as the 
    position vector of the center of buoyancy (CB) relative to the origin of the body frame and having the origin of $\{B\}$ at the CG,
    the restoring forces and moments can be expressed as:
    \begin{equation}
        \mathbf{g}_{\eta} = \begin{bmatrix}
            {}^I_B R^T(\eta)\,[0,0,W-B]^T\\
            - r_b\times {}^I_B R^T(\eta_2)\,[0,0,B]^T
        \end{bmatrix}.
        \label{eq:restoring_forces}
    \end{equation}
    Here, $W$ is the weight of the vehicle, $B = \rho_{seawater} g V$ is the buoyant force ($V$ is the volume discplacement of the AUV), and ${}^I_B R(\eta_2)$ is the rotation matrix from the body-fixed frame to the inertial frame.
    \item $\mathbf{\tau}_{dist}$ includes unmodeled external disturbances such as waves (not important in my application) and currents.
\end{itemize}
So, the equation \ref{eq:6dof_dyn_simple} can be rewritten as:
\begin{equation}
    \mathbf{\tau}_{RB} = \mathbf{\tau} - M_A\dot{\mathbf{\nu}}-C_A(\mathbf{\nu})\mathbf{\nu} - D(\mathbf{\nu})\mathbf{\nu} - \mathbf{g}(\mathbf{\eta}) + \mathbf{\tau}_{dist}
    \label{eq:6dof_tau}
\end{equation}
The complete dynamic model, neglecting the disturbances, becomes:
\begin{equation}
    (M_{RB} + M_A)\dot{\mathbf{\nu}} + (C_{RB}(\mathbf{\nu}) + C_A(\mathbf{\nu}))\mathbf{\nu} + D(\mathbf{\nu})\mathbf{\nu} + \mathbf{g}(\mathbf{\eta}) = \mathbf{\tau},
\end{equation}
Or more compactly:
\begin{equation}
    M\dot{\mathbf{\nu}} + C(\mathbf{\nu})\mathbf{\nu} + D(\mathbf{\nu})\mathbf{\nu} + \mathbf{g}(\mathbf{\eta}) = \mathbf{\tau},
    \label{eq:6dof_dyn_complete}
\end{equation}
By assuming that the origin of the body-fixed reference frame coincides with the vehicle's center of gravity, that the body axes are aligned with the principal axes of inertia,
that the added mass matrix $M_A$ is symmetric and positive definite, and that hydrostatic stability conditions hold, the overall inertia matrix $M$ is symmetric and 
positive definite. Furthermore, the damping matrix $D(\mathbf{\nu})$ is positive definite, while the Coriolis-centripetal matrix $C(\mathbf{\nu})$ 
can be parameterized to be skew-symmetric.\\
When the matrix form of the dynamic model is expanded, it yields the full set of equations corresponding to the six degrees of freedom:
\begin{equation}
    \begin{split}
        m_u\dot{u} - m_vvr + m_wwq + d_uu = \tau_u \\
        m_v\dot{v} + m_uur + m_wwp + d_vv = \tau_v \\
        m_w\dot{w} - m_uuq + m_vvp + d_ww = \tau_w \\
        m_p\dot{p} - m_{vw}vw - m_{qr}qr + d_pp + z_bBcos(\theta)sin(\phi) = \tau_p \\
        m_q\dot{q} + m_{uw}uw + m_{pr}pr + d_qq - z_bBsin(\theta) = \tau_q \\
        m_r\dot{r} - m_{uv}uv - m_{pq}pq + d_rr = \tau_r
    \end{split}
    \label{eq:6dof_expanded}
\end{equation}
Where the following parameters are defined:\\
\begin{minipage}[t]{0.3\textwidth}
\begin{align*}
    m_u &= m - X_{\dot{u}} \\
    m_v &= m - Y_{\dot{v}} \\
    m_w &= m - Z_{\dot{w}} \\
    m_p &= I_x - K_{\dot{p}} \\
    m_r &= I_y - M_{\dot{q}} \\
    m_r &= I_z - N_{\dot{r}}
\end{align*}
\end{minipage}
\hfill
\begin{minipage}[t]{0.3\textwidth}
\begin{align*}
    m_{uv} &= m_u - m_v \\
    m_{uw} &= m_u - m_w \\
    m_{vw} &= m_v - m_w \\
    m_{pq} &= m_p - m_q \\
    m_{pr} &= m_p - m_r \\
    m_{qr} &= m_q - m_r
\end{align*}
\end{minipage}
\hfill
\begin{minipage}[t]{0.3\textwidth}
\begin{align*}
    d_u &= -X_u - X_{|u|u}|u| \\
    d_v &= -Y_v - Y_{|v|v}|v| \\
    d_w &= -Z_w - Z_{|w|w}|w| \\
    d_p &= -K_p - K_{|p|p}|p| \\
    d_q &= -M_q - M_{|q|q}|q| \\
    d_r &= -N_r - N_{|r|r}|r|
\end{align*}
\end{minipage}

\section{BlueRov2 AUV}
The BlueROV2 is a commercial ROV (Remotely Operated Vehicle) developed by Blue Robotics. It is an open-source, modular system that is widely used in both academic 
and industrial settings for research, testing, and inspection applications. Its frame is built from high-strength anodized aluminum and plastic components, 
ensuring durability while keeping the overall weight, $11.5 \text{ Kg}$, light. The open-frame structure allows easy integration of additional 
sensors, payloads, and modifications.

The vehicle is equipped with six thrusters (Blue Robotics T200) arranged in a vectored configuration. Four thrusters are mounted on the horizontal plane, 
oriented at $45^\circ$ with respect to the vehicle axes, allowing precise control of surge, sway, and yaw. The remaining two thrusters are mounted vertically, enabling 
heave, pitch, and roll control. This configuration provides actuation over all six degrees of freedom, making the BlueROV2 a fully actuated underwater vehicle.

The vehicle's buoyancy and stability are provided by syntactic foam blocks mounted on the upper part of the frame, while ballast weights are attached at the bottom to 
lower the center of gravity. Knowing the mass of the vehicle, the volume $V = 0.011054$, the density of water $\rho_{seawater} = 1028$ and $r_b = [0, 0, 0.0420]^T$, 
we can easily compute the restoring component.

Further technical details about the BlueROV2 and its components can be found in the official datasheets provided by the manufacturer \href{https://bluerobotics.com/wp-content/uploads/2017/03/br_bluerov2_datasheet_rev3-bleed-dragged.pdf}{BlueRov2 datasheet} 
and \href{https://cdn.robotshop.com/media/b/blu/rb-blu-27/pdf/rb-blu-27_-_documentation.pdf}{T200 datasheet}.\\
The coefficients of the BlueRov2 model used are derived from the identification work carried out by DSOR-ISR, they are described in the following table:

\begin{table}[H]
    \centering
    \begin{tabular}{l c c c} \toprule
        {Inertia} & {Added mass} & {Linear damping} & {Quadratic Damping} \\ \midrule
        $I_x = 0.21$ & $X_{\dot{u}} = -27.08$ & $X_u = -0.1213$  & $X_{|u|u} = -23.9000$ \\
        { } & $Y_{\dot{v}} = -25.952$ & $Y_v = -1.1732$  & $Y_{|v|v} = -46.2700$ \\
        $I_y = 0.245$ & $Z_{\dot{w}} = -29.9081$ & $Z_w = -1.1130$  & $Z_{|w|w} = -50.2780$ \\
        { } & $K_{\dot{p}} = -1$ & $K_p = -0.5$  & $K_{|p|p} = -1$ \\
        $I_z = 0.245$ & $M_{\dot{q}} = -1$ & $M_q = -0.5$  & $M_{|q|q} = -1$ \\
        { } & $N_{\dot{r}} = -1$ & $N_r = -0.5$  & $N_{|r|r} = -1$ \\ \bottomrule
    \end{tabular}
    \caption{Added mass, linear damping and quadratic damping coefficients of the BlueRov2 robot}
    \label{tab:BlueRov_param}
\end{table}

% descrivere tutti i parametri del BlueRov seguendo per filo e per segno quello fatto nella tesi per il medusa
% Alla fine precisare che il BlueRov2 non è AUV, ma viene considerato da me come tale per semplicità avendo il modello preciso
% e dovendo utilizzare un robot con totale attuazione
Although the BlueROV2 is originally designed as a Remotely Operated Vehicle (ROV), in this work it is modeled and employed as an Autonomous Underwater Vehicle (AUV). 
This choice is justified by the fact that the platform is fully actuated in all six degrees of freedom, with independent thrusters for each motion, making it a 
suitable candidate for testing the proposed control and estimation algorithms in simulation. In the following sections, the BlueROV2 will therefore be analyzed as 
an AUV, even though it is not one in practice. The possible hardware and software modifications required to achieve such a transformation in reality are beyond the 
scope of this work and will not be addressed.

\section{Summary}
In this chapter the fundamental equations governing the dynamics of an underwater vehicle have been presented highlighting the properties concerning the structure of the model. The conventions used in marine robotics to define
the reference frames and the main parameters have been illustrated. The general equations of dynamics for a vehicle in 6 degrees of freedom have been described
in detail, highlighting the contributions of each term. Finally, the BlueRov2 vehicle has been introduced, and the parameters used in the simulation have been 
derived from an identification study conducted by DSOR-ISR.


