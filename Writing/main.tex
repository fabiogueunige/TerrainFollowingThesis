
\documentclass[12pt,a4paper]{article}

% Pacchetti essenziali
\usepackage[utf8]{inputenc}
\usepackage[T1]{fontenc}
\usepackage[italian]{babel}
\usepackage{geometry}
\usepackage{url}
\usepackage{hyperref}

% Configurazione pagina
\geometry{margin=2.5cm}

% Configurazione hyperref per i link
\hypersetup{
    colorlinks=true,
    linkcolor=blue,
    urlcolor=blue,
    citecolor=blue
}

\title{Note per la Tesi}
\author{Fabio}
\date{\today}

\begin{document}

\maketitle

\section{Problematiche Affrontate}

Useful Chat GPT conversation helping with this following problems:\\
\begin{itemize}
    \item Vantaggio della mia tesi e utilità scientifica\\
    \item Come controllare surge e heave insieme (non ottima risposta)\\
    \item Gestire discontinuità degl'angoli terreno e robot -> usare quaternioni\\
    \item Come gestire i valori dei sensori che non arrivano o arrivano sbagliati -> pesare i valori\\
    \item Gestire n e s paralleli -> sempre dare pesi diversi e cambiare covarianza\\
\end{itemize}

\href{https://chatgpt.com/share/686d001f-fa28-8004-a8b7-eec4cea88c3a}{Dacci una occhiata}!

\end{document}