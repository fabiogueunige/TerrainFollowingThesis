\documentclass[12pt,a4paper]{article}

% Pacchetti essenziali (sistemare)
\usepackage[utf8]{inputenc}
\usepackage[T1]{fontenc}
\usepackage{geometry}
\usepackage{url}
\usepackage{hyperref}
\usepackage{graphicx} % Required for inserting images
\usepackage{amsmath}

% Configurazione pagina
\geometry{margin=2.5cm}

% Configurazione hyperref per i link
\hypersetup{
    colorlinks=true,
    linkcolor=blue,
    urlcolor=blue,
    citecolor=blue
}

\title{Note per la Tesi}
\author{bau bau}
\date{\today}

\begin{document}

\maketitle

\section{Appunti veloci}

Useful Chat GPT conversation helping with this following problems:\\
\begin{itemize}
    \item Vantaggio della mia tesi e utilità scientifica\\
    \item Come controllare surge e heave insieme (non ottima risposta)\\
    \item Gestire discontinuità degl'angoli terreno e robot -> usare quaternioni\\
    \item Come gestire i valori dei sensori che non arrivano o arrivano sbagliati -> pesare i valori\\
    \item Gestire n e s paralleli -> sempre dare pesi diversi e cambiare covarianza\\
\end{itemize}

\href{https://chatgpt.com/share/686d001f-fa28-8004-a8b7-eec4cea88c3a}{Dacci una occhiata}!
\\
\section{Abstract}
Qui scrivere abstract finale andando a ricoprire tutto ciò che è stato cstruito per
la tesi\\
\section{Introduction}
% What i have to write about:
% 1) Descrivere importanza degli auv 
% 2) Descrivere perchè è importante che gli auv siano a guida autonoma
% 3) addentrasi sulla esplorazione del fondale marino e la difficoltà di questa
% 4) Descrivere i problemi di navigazione e controllo degli auv
% 5) Descrivere come non esista una navigazione che segua il profilo del terreno
% 6) approfondire i diversi sitemi di navigazione e controllo degli auv diversi dal mio
% 6) Descrivere come siano sfide nell'ambiente marino (qualità dei dati batimetrici, Robustezza in ambienti complessi, Ottimizzazione energetica)
% 6) Descrivere come la mia tesi risolve questo problema focalizzandosi su (metodi di navigazione avanzati e integrazione sensoriale)
% 7)
    
\section{Methods}
% Concentrarsi su:
% 1) Descrivere il metodo di navigazione e controllo proposto
% 2) Descrivere come il metodo affronta le sfide della navigazione e del controllo degli AUV
% 3) Descrivere come il metodo integra i dati dei sensori e le tecniche di filtraggio
% 4) Descrivere come il metodo migliora l'efficienza energetica e la robustezza
% 5) Descrivere come il metodo può essere implementato in un AUV reale
% 6) Descrivere come il metodo può essere applicato in scenari reali
% 7) Descrivere come il metodo può essere esteso o migliorato in futuro

\section{Conclusion}
% Concentrarsi su:
% 1) Riepilogo dei risultati ottenuti
% 2) Implicazioni per la comunità scientifica e per l'industria
% 3) Trovare vantaggi e svantaggi del metodo per l'industria (es: Mappatura batimetrica di precisione, Ispezione e manutenzione subacquea, Monitoraggio ambientale, Ricerca oceanografica e archeologia subacquea)
% 4) Limitazioni dello studio
% 5) Prospettive future e direzioni di ricerca

% Questo approccio non solo migliora le prestazioni operative di un AUV, ma apre la strada a missioni più lunghe, sicure ed efficienti, con un impatto significativo in applicazioni scientifiche, industriali e ambientali.




\end{document}