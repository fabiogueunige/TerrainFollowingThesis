\chapter{Sensor}
\label{chap:third}
\ifpdf
    \graphicspath{{Chapter3/Figures/PNG/}{Chapter3/Figures/PDF/}{Chapter3/Figures/}{Chapter3/Figures/EPS/}}
\else
    \graphicspath{{Chapter3/Figures/EPS/}{Chapter3/Figures/}}
\fi

% Before a fast overview di quello che affronta il capitolo
This chapter highlights the critical role of sensors in marine robotics, with particular emphasis on their relevance to autonomous 
navigation. It outlines the key challenges that arise in this context and examines the specific sensors employed to implement the 
navigation algorithm.\\
The discussion is limited to those sensors directly involved in navigation, the AHRS, the DVL, and the echosounder, 
while excluding auxiliary sensors responsible for monitoring system status or ensuring the proper functioning of other subsystems. 
Each of the selected sensors is analyzed in terms of its purpose, underlying physical principles, and integration within the AUV.\\ 
The chapter concludes with a discussion on their combined use and potential future extensions through the addition of other sensing 
technologies.

\section{Sensor importance?????}
Sensory systems are fundamental to the autonomous operation of Autonomous Underwater Vehicles (AUVs), enabling precise navigation, 
mapping of the marine environment, and acquisition of scientific data in underwater operating conditions. The underwater 
environment presents unique challenges for sensors, including the absence of GPS signals, attenuation of electromagnetic 
communications, and poor visibility conditions.

The sensor configuration of a modern AUV typically integrates three main categories of sensors: attitude and heading reference 
systems (AHRS), Doppler velocity measurement systems (DVL), and echosounder arrays for bathymetry and seabed mapping.

This sensory combination provides the inertial navigation, the estimation of the vehicle's velocity relative to the inertial 
reference frame, and surrounding environment perception capabilities necessary for autonomous underwater operation. 
The integration of these sensors requires sophisticated data fusion algorithms, such as the Kalman Filter (addressed in \ref{chap:fourth}), 
to combine information from different sensor sources, compensating for the individual limitations of each system and improving the 
overall accuracy of navigation and environmental perception.

% Sistema i nomi
\section{AHRS}
\section{DVL}
\section{Echosounder}
\section{Possible integration}
% non saprei se metterla. Sarebbe per agiungere sensori nel futuro
\section{Summary}