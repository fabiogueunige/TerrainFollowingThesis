\chapter{Sensor}
\label{chap:third}
\ifpdf
    \graphicspath{{Chapter3/Figures/PNG/}{Chapter3/Figures/PDF/}{Chapter3/Figures/}{Chapter3/Figures/EPS/}}
\else
    \graphicspath{{Chapter3/Figures/EPS/}{Chapter3/Figures/}}
\fi

% Before a fast overview di quello che affronta il capitolo
This chapter highlights the critical role of sensors in marine robotics, with particular emphasis on their relevance to autonomous 
navigation. It outlines the key challenges that arise in this context and examines the specific sensors employed to implement the 
navigation algorithm.\\
The discussion is limited to those sensors directly involved in navigation, the AHRS, the DVL, and the echosounder, 
while excluding auxiliary sensors responsible for monitoring system status or ensuring the proper functioning of other subsystems. 
Each of the selected sensors is analyzed in terms of its purpose, underlying physical principles, and integration within the AUV.\\ 
The chapter concludes with a discussion on their combined use and the advantages of sensor fusion in enhancing the overall 
performance of the navigation system.

\section{Sensors for dead reckoning}
% Potrei spostare in fondo e fare discorso su quello che ho usato io per BlueRov2
Sensory systems are fundamental to the autonomous operation of Autonomous Underwater Vehicles (AUVs), enabling precise navigation, 
mapping of the marine environment, and acquisition of scientific data in underwater operating conditions. The underwater 
environment presents unique challenges for sensors, including the absence of beacons and GPS signals, attenuation of electromagnetic 
communications, and poor visibility conditions. Moreover, the unknown environment doesn't allow for the use of pre-existing maps, 
making real-time perception and navigation essential.

The sensor configuration for autonomous navigation of a modern AUV typically integrates three main sensors: attitude and heading reference 
systems (AHRS), Doppler velocity measurement systems (DVL), and echosounder arrays for seabed scanning (or a camera). This sensory combination provides the 
inertial navigation, the estimation of the vehicle's velocity relative to the inertial 
reference frame, and surrounding environment perception capabilities necessary for autonomous underwater operation. 
Each sensor requires sophisticated data fusion algorithms, such as the Kalman Filter (addressed in \ref{chap:fourth}), to reduce the inherent noise and
errors in their measurements. Moreover, in many cases, it is necessary to combine information from different sensor sources, compensating for the individual 
limitations of each system and improving the overall accuracy of navigation and environmental perception. Indeed, when for a long time the AUV does not receive 
his absolute position, the error in the position estimate tends to grow unbounded, a phenomenon known as drift. One simple solution, but not the most accurate, 
is to fuse the AHRS and DVL data to estimate the AUV's position through dead reckoning.


\section{Attitude and heading reference system}
The Attitude and Heading Reference System (AHRS) is a key sensor for determining the orientation of the AUV in three-dimensional space.
It typically fuses data from three types of Micro-Electro-Mechanical Systems (MEMS) to provide accurate estimates of roll, pitch, and yaw angles.\\
Other AHRS technologies exist, such as Fiber Optic Gyroscopes (FOG) or Ring Laser Gyroscopes (RLG), which offer higher precision but are more expensive and primarily used in 
specialized applications. MEMS technology, being standard for the BlueRov2, will be the focus here.\\
These MEMS-based AHRS typically include \cite{klugaMotionSensorsData2024}:
\begin{itemize}
    \item Triaxial gyroscope: Measures angular velocities around the three axes, capturing rapid changes in orientation.
    \item Triaxial accelerometer: Measures linear accelerations along the three axes, which can be used to estimate roll and pitch angles.
    \item Triaxial magnetometer: Measures the magnetic field strength along the three axes, providing heading information relative to the Earth's magnetic field.
\end{itemize}
The physical principles of these sensors differ. The gyroscope uses the Coriolis effect to measure the motion of internal proof masses proportional to the angular 
velocity; however, it is prone to drift over time due to temperature variations and aging, and it cannot provide absolute orientation or heading information. 
The accelerometer relies on Newton's second law of motion to measure linear acceleration by sensing the direction of gravity, but its readings can be affected by dynamic accelerations. 
The magnetometer used in the BlueRov2 is an Anisotropic Magneto-Resistance (AMR) sensor, which exploits the tendency of certain materials (typically ferromagnetic) to change 
their electrical resistance in response to the Earth's magnetic field. However, it is more susceptible to noise induced by the AUV's structure. The magnetic field measurement's bias is hard to 
estimate \cite{troniMagnetometerBiasCalibration2014} because the bias interface depend on the environmental aspect, which are varying both with time and space in all directions, 
therefore it requires careful calibration and filtering \cite{koSineRotationVector2016}.\\
The AHRS combines the data from these three sensors using sensor fusion algorithms, such as the Extended Kalman Filter (EKF) or complementary filters, to provide a robust and
accurate estimate of the AUV's orientation. The gyroscope data is used for short-term orientation changes, while the accelerometer and magnetometer data are used to correct 
for drift and provide absolute orientation references. The roll and pitch angles are primarily derived from the accelerometer readings, which are generally more 
reliable than the magnetometer data.\\
The acquisition of the yaw angle depends primarily on the magnetometer readings. Consequently, due to noise, the yaw estimate is generally less accurate than 
the roll and pitch angles \cite{koComparisonAttitudeEstimation2016}. In this work, this is not a major issue because the yaw angle is mainly relevant for the 
path-following node and not for the terrain-following node. However, it is still actuated to generate a simple trajectory to test the algorithm performance.\\ 
To improve the reliability of the heading, it is often necessary to fuse the AHRS data with the DVL measurements to obtain a more accurate yaw angle estimation.  
For completeness, it is worth noting that more precise magnetometers, such as Fluxgate sensors, exist but are also significantly more expensive and may not always 
fit within the AUV's design constraints \cite{timmermannComparisonNoiseLevels2025}.

% magari potrei togliere discorso frequenze di funzionamento
AHRS measurements are typically provided at frequencies ranging from 50 to 200 Hz, delivering real-time orientation data to the AUV's control system. The optimal sampling 
frequency is generally between 200$Hz$ and 400$Hz$, with a bandwidth of 10$Hz$ to 50$Hz$, to ensure reliable acquisition at the AUV's operating rate.

The AHRS is essential for stabilizing vehicle motion and achieving precise navigation, but its integration in an AUV requires specific considerations.  
It should be mounted near the AUV's center of gravity to minimize the influence of dynamic linear accelerations and isolated from vibrations and shocks that 
could degrade sensor readings.

\section{Doppler Velocity Log (DVL)}
The Doppler Velocity Log (DVL) is a crucial sensor for measuring the relative velocity of an AUV. 
It relies on the Doppler effect, which occurs when the frequency of a wave appears altered due to the relative movement between the source emitting the wave and the 
observer receiving it. In fact, the DVL transmits acoustic pulses along multiple beams, directed in different directions, and measures the frequency shift of the 
echoes reflected by the receivers. By analyzing these frequency shifts, the DVL can calculate the velocity of the AUV, relative to the receiver, following this formula:
\begin{equation}
    \Delta f = \frac{2 f_0 v cos(\theta)}{c}
    \label{eq:doppler}
\end{equation}
where $\Delta f$ is the frequency shift, $f_0$ is the emitted frequency, $v$ is the three dimensional relative velocity of the AUV, $\theta$ is the angle between the acoustic beam and 
the direction of motion, and $c$ is the speed of sound in water \cite{annapurnaEnhancingAccuracyDoppler2024}.

This device usually employs four acoustic beams in a Janus configuration, which allows for the measurement of velocity in three dimensions. Each beam is 
oriented at a specific angle to the vertical axes, typically around 20 to 30 degrees, to measure:
\begin{itemize}
    \item Forward and backward velocity components (surge)
    \item Lateral velocity components (sway)
    \item Vertical velocity components (heave)
\end{itemize}
Considering the formula \ref{eq:doppler} we can focus on three main factors that influence the DVL performance. The speed of the sound in water $c$ 
is affected by environmental conditions such as temperature, salinity, and pressure. This variability can introduce errors in velocity measurements if not properly compensated.
The angle $\theta$ of the acoustic beams is critical for accurate velocity estimation. Misalignment or calibration errors can lead to significant inaccuracies, especially in the 
lateral and vertical velocity components.
In the end the emitted frequency $f_0$ also plays a role, as higher frequencies provide better resolution but are more susceptible to attenuation in water, limiting the effective range of the DVL.
instead, lower frequencies can travel further but may offer less precise measurements. There are different frequency choices for DVLs, like 600$kHz$ for short-range and high accuracy applications, or
38 $kHz$ for long-range applications with lower accuracy \cite{sarangapaniMultifrequencyPhasedArray2022}.

There are two operating modes of DVL: the bottom-tracking mode (DVL-bt) and the water-tracking mode (DVL-wt). The bottom-tracking mode measures the velocity of the AUV relative to the seabed. 
The accuracy of the data depends also on the seabed material composition and geometry, as soft sediments can absorb acoustic signals, or some acoustic beams may be lost, leading to weaker 
echoes and less reliable velocity measurements. However, When operating in mid-water zone, the DVL may lose bottom track due to the limited sensor range. In this case, its measurements will be affected by sea currents \cite{liuSINSDVLIntegrated2022}.
This is the water-tracking mode, which measures the velocity relative to the water mass, knowing that $v_{water} = v_{AUV} - v_{current}$.\\
It is possible thanks to the presence of small particles or plankton in the water that reflect the acoustic signals, but only using high frequencies. However, this mode is generally less accurate 
than bottom-tracking, as the water mass can have its own movement due to currents, which estimation is not always available.

% Fare discorso su BLUEROV2
% Penso a Lisbona usino Teledyne Explorer DVL perchè solo wt mode
% Ho trovato rif solo per WaterLinked DVL A50 che sarebbe meglio perchè lo scopo è di rimanere molto vicino al fondale
In the BlueRov2 .... {scrivere}

\section{Echosounder}

\section{Summary}